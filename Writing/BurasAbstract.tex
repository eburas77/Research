\documentclass{article}

\usepackage{graphicx}
\usepackage{hyperref}
\usepackage{listings}
\usepackage{color}
\usepackage{verbatim}
\usepackage{amsmath}
\usepackage{setspace}



\begin{document}
\title{A Multigrid Solver for Graph Laplacian Linear Systems on Power-Law Graphs}
\author{Eric D. Buras}


\maketitle
\begin{center}
Adviser: Dr. Matthew Knepley\\
In Partial Fulfillment of the Master's Degree in the Computational and Applied Mathematics Department Rice University\\
\end{center}
\vspace{15 mm}
\section{Abstract}
\begin{spacing}{1}
The Laplacian matrix, $L$, of a graph, $G$, contains degree and edge information of a given network. Solving a Laplacian linear system $Lx = b$ provides information about flow through the network, and in specific cases, how that information orders the nodes in the network. I propose a novel way to solve this linear system by first partitioning $G$ into its maximum locally-connected subgraph and a small subgraph of the remaining teleportation edges. I then apply optimal multigrid solves to the locally-connected subgraph, and linear algebra and a solve on the so-called "teleportation" subgraph to solve the original linear system. I show results for this method on real-world graphs from the biological systems of the \textit{C. Elegans} worm, Facebook friend networks, and the power grid of the Western United States.
\end{spacing}

%\bibliographystyle{siam}
%\bibliography{mastersbib}


\end{document}
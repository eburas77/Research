\documentclass{article}

\usepackage{graphicx}
\usepackage{hyperref}
\usepackage{listings}
\usepackage{color}
\usepackage{verbatim}


\begin{document}
Eric Buras 2/19/16 Literature Review 4
\\
\begin{itemize}
\setlength\itemsep{1em}
\item{\textbf{Complete Citation:}}
\\
D. J. Watts and S. H. Strogatz, Collective dynamics of 'small-world' networks, \textit{Nature} \textbf{393}, 440-442.

\item{\textbf{Key Words:}} 
\\
Small World Networks, Social Networks, Six Degrees of Separation.

\item{\textbf{General Subject}:}
\\
Graph Theory and Application.

\item{\textbf{Specific Subject:}}
\\
Properties of 'small world' networks. Real-world graphs that can be modeled by these small world networks.

\item{\textbf{Authors' Hypothesis or Claim:}}
\\
The authors show that simple models of networks that are rewired with increasing amounts of disorder can be modeled by 'small world' networks. These networks can be highly clustered, yet have relatively small characteristic path length. The authors apply this model to the C. Elegans neural network, the power grid of the western United States, and the Collaboration graph of film actors.

\item{\textbf{Methodology:}}
\\
Graph theory and associated numerical and analytic proofs.

\item{\textbf{Result(s):}}
\\
\begin{itemize}
\item
This small world network model has short characteristic path length and high clustering coefficient.
\item
Characteristic path length and cluster coefficient computed for C. Elegans, power grid, and film actor graphs.
\item
Functional significance of small-world network on real dynamical system: infectious disease spread. Calculated time spread of disease is much shorter than anticipated. Our small-world network society is very susceptible to disease event.

\end{itemize}
\item{\textbf{Evidence:}}
\\
Attributes of three graphs found using standard graph computation. Disease spreading found using dynamical system simulation.
\item{\textbf{Summary of Key Points:}}
\\
The proposed small-world network can be found in many natural networks, including biological systems, social interactions, and manufactured societal creations. The author's have a better model for studying these.
\item{\textbf{Context and Relationships:}}
\\
Much work has been done to study random graphs, and structured lattice grids, this work combines the two. Also, much work has been done to study dynamical system simulation. The authors propose a way to introduce graph theory into this distinct field.

\item{\textbf{Significance:}}
\\
This paper is significant for my work in that I use Fan Chung's algorithm to find max locally connected subgraphs in small world networks. She also proposes bounds on average diameter and distance for such graphs that are studied in this paper. This paper introduces me to the C. Elegans neural network graph which I am using in my work.
\item{\textbf{Important Figures and/or Tables:}}
\\
Table 1 gives empirical calculations of characteristic path length and clustering coefficient for the three networks. Figure 3 shows the results of the infectious disease dynamical system simulation.

\item{\textbf{Cited References to Follow Up On:}}
\\
\begin{itemize}
\item Milgram, S. The small world problem. \textit{Psychol. Today} \textbf{2}, 60-67 (1967).
\item Wasserman, S. and Faust, K. \textit{Social Network Analysis: Methods and Applications} (Cambridge Univ. Press, 1994).
\end{itemize}
\end{itemize}

\end{document}
